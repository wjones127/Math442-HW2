From the assignment\+:

\begin{quote}
The main goal of this exercise is to empirically determine the sizes of the different cache layers in a processor by building a micro-\/benchmark that measures memory performance for different request sizes. \end{quote}


\subsection*{Part 1\+: Reading one byte from N-\/size buffer}

Our first goal is the test the latency of reading one byte from a buffer of size {\ttfamily N}, where we are varying {\ttfamily N}. Our approach will be to initialize an array of random numbers, and then run a loop where we generate a random index and measure the time it takes to access the index.

The overhead of measuring time can mean that we are mostly measuring the amount of time to compute current time, rather than time to access. At the same time, simply taking one average per array size could not give us an accurate sense of the variance. So we will try bucketing the computations; e.\+g. measure the time to access {\ttfamily k} different random bytes in the array.

\subsubsection*{Notes}

This is really difficult to measure in a way that gets us a really clear drop-\/off in performance. The more iterations in a bucket, the more misses will just be smoothed out. But with fewer iterations in a bucket, we just end up measuring the time it takes to get the time. 